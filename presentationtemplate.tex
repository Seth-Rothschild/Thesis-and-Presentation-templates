\documentclass{beamer}
%% +---------------------+ 
%% | Begin Customization | 
%% +---------------------+ 

\usetheme[titleformat=smallcaps, numbering=none, block=fill]{metropolis}
% Metropolis theme is included in texlive2016
% More info (and a manual!) at https://github.com/matze/mtheme 

% A handy customization guide is 
% http://www.cpt.univ-mrs.fr/~masson/latex/Beamer-appearance-cheat-sheet.pdf

% The beamer manual can be found at
% http://www.tuteurs.ens.fr/noncvs/docs/beamer/beameruserguide.pdf

% Here are some nice colors to use
% Found in Tufts branding guide
% http://communications.tufts.edu/wp-content/uploads/Tufts_Branding_Guidelines.pdf

\definecolor{classicblue}{RGB}{62,142,222}
\definecolor{betterblue}{RGB}{49,114,174}
\definecolor{classicbrown}{RGB}{81,44,29}
\definecolor{accessibleblue}{RGB}{52,96,148}

\definecolor{extendedgrey}{RGB}{100,100,105}
\definecolor{extendedorange}{RGB}{243,138,0}
\definecolor{extendeddarkorange}{RGB}{212,93,0}
\definecolor{extendedpurple}{RGB}{80,7,120}
\definecolor{extendedfadedblue}{RGB}{127,169,174}
\definecolor{extendedteal}{RGB}{0,176,185}
\definecolor{extendeddarkgreen}{RGB}{86,108,17}

% General beamer colors
\setbeamercolor*{palette primary}{fg=white,bg=accessibleblue}
\setbeamercolor*{palette secondary}{fg=white,bg=accessibleblue}
\setbeamercolor*{palette tertiary}{fg=white,bg=accessibleblue}
\setbeamercolor*{palette quaternary}{fg=accessibleblue,bg=accessibleblue}

% Define colors for theorem blocks
%\setbeamercolor{block title}{use=structure,fg=white,bg=extendedteal}
\setbeamercolor{block body}{parent=normal text,use=block title,bg=block title.bg!10!bg}

% Metropolis specific colors
\setbeamercolor{normal text}{fg=extendedgrey}
\setbeamercolor{alerted text}{fg=classicblue}
\setbeamercolor{example text}{fg=classicblue}
\setbeamercolor{title separator}{fg=classicblue}
\setbeamercolor{progress bar in head/foot}{fg=classicblue}
\setbeamercolor{progress bar in section page}{fg=classicblue}

%% +-----------------------+
%% |   End Customization   |
%% +-----------------------+

%% +--------------+
%% |  Begin Math  |
%% +--------------+
 
\usepackage{amsmath}  % Standard AMS packages
\usepackage{amssymb}
\usepackage{amsbsy}

\usepackage{url}      % Linking slides
\usepackage{hyperref}

\usepackage{tikz-cd}  % Tikz exact sequences
\usepackage{appendixnumberbeamer} % Hide slide numbers after appendix

\renewcommand{\thefootnote}{}

\newcommand\scalemath[2]{\scalebox{#1}{\mbox{\ensuremath{\displaystyle #2}}}}
% Scale large pictures

\newtheorem{proposition}{Proposition}
\newtheorem{application}{Application}

\title[Title in footer]{Title}
\author[Name in footer]{Firstname M. Lastname}
\date[Date in footer]{March 30, 2017}
\institute[Tufts University]{Tufts University\\Department of Mathematics}

\begin{document}
\begin{frame}[plain] % No header
  \frametitle{}
\titlepage
\end{frame}

\begin{frame}
  \frametitle{Quotes}
 \begin{block}{}
   {\large ``This is a well sized block quote. The
     \emph{block body} color is defined above''

     }  \vskip5mm \hspace*\fill{\small---
     Seth Rothschild}
 \end{block}
\end{frame}

\begin{frame}
  \frametitle{Some Tikz}
  \center{\begin{tikzpicture}
\draw (0,0) circle (3cm);
\draw[fill=white] (0,3) circle [radius=0.5] node {0};
\draw[fill=white] (2.85,0.927) circle [radius=0.5] node {1};
\draw[fill=white] (1.76,-2.43) circle [radius=0.5] node {2};
\draw[fill=white] (-1.76,-2.43) circle [radius=0.5] node {3};
\draw[fill=white] (-2.85,0.927) circle [radius=0.5] node {4};
  \end{tikzpicture}}
\end{frame}
\begin{frame}
\frametitle{The ``visible'' overlay is nice}
  You can define \alert{overlays} in beamer 

  \visible<2->{Maybe you want it to preserve spacing}

  \only<3>{Maybe you don't}
  
  \invisible<1-3>{Overlays can be applied to all kinds of things}
\end{frame}


\section{The progress bar shows here}
\begin{frame}
\label{hiddenslide1-origin}
\frametitle{Hidden slides}
  \hyperlink{hiddenslide1}{Notice} that this is the last counted section according to
  beamer.
  
  \begin{enumerate}
  \item<2-> We've hidden some slides after the end of the
    talk
  \item<3-> The word ``\hyperlink{hiddenslide1}{Notice}'' is
    a link to that slide.
  \end{enumerate}
\end{frame}
\appendix

\begin{frame}
\visible<1-3>{\titlepage} % Include a couple of titlepages
                          % to hide hidden slides
\end{frame}
\begin{frame}
  \frametitle{Hidden Slide Index}
  I found it helpful to have a list of links here
  \begin{itemize}
  \item \hyperlink{hiddenslide1}{This is a hidden slide}
  \end{itemize}
\end{frame}
\begin{frame}[label=hiddenslide1]

  \frametitle{This is a hidden slide!}
  \hyperlink{hiddenslide1-origin}{It} is worthwhile
  to link back to the origin so you can go back and forth. 
\end{frame}
\end{document}

%%% Local Variables:
%%% mode: latex
%%% TeX-master: t
%%% End:
